

%-----------------------------------------------------------------------------------------%
%-----------------------------------------------------------------------------------------%

\chapter{Preface: Sex and/or Gender?}

Sex is a biological concept -- based on chromosomal, cellular, hormonal, 
and molecular characteristics.\citep{clayton2016reporting} In contrast 
to this, gender is a social concept -- the reflection of behavioral 
and cultural factors which contribute to both a person's self-identity 
and societal structures by defining norms, roles, expectations, and 
power relations.\citep{clayton2016reporting} Since being introduced 
nearly half a century ago, this distinction has become widely accepted 
in the medical sciences and is now visible within journals editorial 
policies on how sex and gender should be reported, for example in The 
Lancet.\citep{schiebinger2016editorial} Although sex is a key variable 
in most studies and models of the health sciences, it seems that there 
is still no clear standard for how gender can be measured and followed 
throughout all stages of research.\citep{schiebinger2016gender} This 
issue was ever-present while working on the thesis, and I discovered 
that even the most contemporary research often uses sex and gender 
inconsistently, interchangeably, or inappropriately. Sex and gender 
are distinct concepts and should not be considered synonyms.

The question of whether the term sex and/or gender should be used in 
a research project is an important empirical question -- rather than a 
matter of belief -- as both concepts, one, or neither concept may play 
a key role.\citep{krieger2003genders} For example: Sex differences in 
hormonal levels have an impact on the age at onset of cardiovascular 
diseases.\citep{eskes2007women} In addition, gender-patterned health 
behaviors such as smoking, drinking, or nutrition are at least of 
equal importance for explaining why men experience cardiovascular 
diseases earlier in life than women.\citep{mosca2011sex} However, 
factors not related to sex or gender, such as high levels of air 
pollution, may also be important for explaining increased population-level 
incidence rates.\citep{cosselman2015environmental}

The title of the thesis contains the phrase gender -- not sex. 
However, some sections of the thesis refer to the term sex. 
The Scandinavian languages, including Danish, do not force a distinction 
between sex and gender but use \textit{k{\o}n}, \textit{k\"on}, or 
\textit{kj{\o}nn} for both concepts.\citep{widerberg1998translating}
Underlining this, the journal articles describing the Danish registries 
translate the dichotomous-coded variable \textit{k{\o}n} interchangeably 
as sex and gender -- even within the same publication.\citep{pedersen2011,
sahl2011danish} Additionally, official data and material provided 
by Statistics Denmark often refer to both sex and gender. 

Another important feature emerges from the binary coding of the 
sex/gender variable in the utilized Danish registers. Given such 
a dichotomous concept, it is impossible to study individuals which 
may identify themselves outside the binary framework of "men" and 
"women". The literature has found that the likelihood of certain 
health outcomes, and the pathways to these health outcomes, may differ 
among individuals who do not identify as either "men" or "women".\citep{richards2016non,
koehler2018genders,rider2018health} In this regard, the majority of 
national agencies and data providers have not yet accounted for the 
diversity of gender identities.

Throughout the thesis, I decided to follow the rule of Clayton and 
Tannenbaum (2016) who argued that, in order to avoid misleading conclusions, 
what should be reported is what has been studied or recorded.\citep{clayton2016reporting} 
In practice, this meant I chose to use the term "gender" in applications 
which primarily targeted or incorporated behavioral features -- and 
preferred to use the term "sex" when no information on behavioral 
features were available or when studies being referenced explicitly 
used this terminology.

%-----------------------------------------------------------------------------------------%
%-----------------------------------------------------------------------------------------%
