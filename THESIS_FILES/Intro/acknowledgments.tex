

%-----------------------------------------------------------------------------------------%
%-----------------------------------------------------------------------------------------%


\chapter{Acknowledgments}

At this point, I would like to thank my supervisors Rune Lindahl-Jacobsen, 
Roland Rau, Anna Oksuzyan, and Kaare Christensen. Throughout my PhD, I was 
always well supervised and guided by this multidisciplinary team. Thank you 
very much for sharing your ideas and thoughts honestly with me. I would like 
to thank Anna for hiring me as your first PhD student. Anna's encouragement 
has been a valuable source of motivation throughout my entire PhD. I would 
like to thank Kaare and Rune for encouraging me to become a PhD student in 
Denmark while giving me the freedom to keep my close connections with Rostock 
and the MPIDR. 

With some years of delay, I would like to thank Roland. In 2011, Roland 
spontaneously called his friend Domas at the MPIDR in Rostock to arrange 
a meeting for me on the very same day to discuss my BA thesis ideas. 
Since then, the MPIDR has always been an amazing place for me to learn 
and develop as a researcher.

An outstanding and very special thanks goes to Ute for your patience and 
excellent conversations. I hope we have not caused too many administrative 
problems within our institute -- although we should be proud of causing at 
least one or two.

I would like to thank my colleague, office mate, and good friend Jen for 
being such a great mentor, adviser, and critical reviewer of my papers and 
this thesis. I am very thankful for all of your help and feedback during 
the last months of my PhD.

I also thank my co-authors for their great contributions throughout the 
four papers of my PhD: Lisbeth Aagaard Larsen, Daniel Schneider, Jutta 
Gampe, Alyson van Raalte, and Pekka Martikainen.

Very warm thanks go to my mother Nadja and my sister Lena for their help 
and understanding. Although I have not been very good (or patient enough) 
in explaining what my actual research is about, you have never questioned 
my enthusiasm and enjoyment. I think this thesis marks the point of no 
return; I won't become a dentist anymore.

I also want to thank my friends Mathias and Laura, who were my 
flatmates throughout the preparatory year of my PhD -- the EDSD 
2015/2016 in Rome. Thank you very much for sharing and discussing 
thoughts, hopes, fears, project ideas, wines, memes, as well as Rcode 
and solutions for the pending assignments. This is also the place to 
say thank you to my friends from the 'Saturday Group': June, Karen, 
Stefan, Kieron, Tim, Ainhoa, Aitor, and Jen. Thank you very much 
for letting me win so many rounds of board games, particularly
'Seven Wonders'. Another very special thanks goes to my friends from 
Rostock, of which most I have known for more than 10 years now. Your 
outstanding company made my amazing days feel even greater and the 
difficult days a little more bearable.

And, finally, a last and very big thanks goes to Rosie. It is my biggest 
pleasure to say thank you for an infinite number of things, but in 
particular, for constantly reminding me of all the big and little things 
in life that matter. Thank you very much for teaching me that compromises 
are acceptable up to a certain point -- except when it's about compromising 
on happiness in life. 


\chapter{Funding}

This thesis was primarily funded by the grant to establish the Max Planck 
Research Group 'Gender Gaps in Health and Survival' at the Max Planck Institute 
for Demographic Research, Rostock. Furthermore, the thesis received funding 
from the Max Planck Odense Center on the Biodemography of Aging (MaxO), the 
Interdisciplinary Center on Population Dynamics at the University of Southern 
Denmark, Odense (CPop), and the Max Planck Society within the framework of 
the project 'On the edge of societies: New vulnerable populations, emerging 
challenges for social policies and future demands for social innovation and 
the experience of the Baltic Sea States (2016-2021)'. 

In addition, the papers of the thesis were supported by the US National 
Institute of Health (P01AG031719, R01AG026786, and 2P01AG031719), 
the VELUX Foundation, the European Research Council (Grant Number 716323), 
the Academy of Finland, and the Odense University Hospital within the 
Framework of the AgeCare Program (Academy of Geriatric Cancer Research). 

%-----------------------------------------------------------------------------------------%
%-----------------------------------------------------------------------------------------%
