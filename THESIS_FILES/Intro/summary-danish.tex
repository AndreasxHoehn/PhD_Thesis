

%-----------------------------------------------------------------------------------------%
%-----------------------------------------------------------------------------------------%


\chapter{Summary -- Danish}

Selvom kvinder gennemg{\aa}ende har et d{\aa}rligere helbred end m{\ae}nd, 
har de mindre d{\o}dsrisiko. Dette er vist i studier p{\aa}  tv{\ae}rs af 
lande og tidsperioder. Alligevel anf{\ae}gtes generaliserbarheden af dette 
paradoks -- is{\ae}r for mennesker som er ovre den reproduktive alder, hvor 
helbredet bliver d{\aa}rligere og aldringsprocesserne accelererer. Rutinem{\ae}ssigt 
indsamlede hospitalsregistreringer giver v{\ae}rdifuld, objektiv 
information og skaber et godt grundlag for at studere k{\o}nsforskelle i 
d{\o}delighed og helbred. Disse oplysninger omfatter medicinsk diagnosticerede 
lidelser, som ofte kan kobles til data vedr{\o}rende d{\o}delighed, og som 
er tilg{\ae}ngelige i en lang r{\ae}kke lande, heriblandt Danmark.

Hospitalsregistreringer for hele den danske befolkning dannede grundlag 
for fire studier, der analyserede k{\o}nsforskelle i helbred og d{\o}delighed 
i forbindelse med hospitalsindl{\ae}ggelser. Det f{\o}rste studie estimerede 
den mandlige overd{\o}delighed, b{\aa}de samlet og {\aa}rsagsspecifik, efter 
den f{\o}rste hospitalsindl{\ae}ggelse efter det fyldte 50 {\aa}r. Resultaterne 
fra det f{\o}rste studie tyder p{\aa} at kvinders mindre d{\o}dsrisiko efter 
den reproduktive alder kan tilskrives det forhold at kvinder har lavere risiko 
for at d{\o} i forbindelse med helbredsforv{\ae}rring.

Det andet studie unders{\o}gte hvordan den k{\o}nsspecifikke behandlingss{\o}gende 
adf{\ae}rd {\ae}ndrede sig i forbindelse med et st{\o}rre helbredsm{\ae}ssigt chok 
defineret som f{\o}rste hospitalsindl{\ae}ggelse efter det fyldte 60 {\aa}r p{\aa} 
grund af enten slagtilf{\ae}lde, hjerteinfarkt, kronisk obstruktiv lungesygdom eller 
mave-tarmkr{\ae}ft. Resultaterne indikerer at kvinders st{\o}rre brug af det prim{\ae}re 
sundhedsv{\ae}sen kan skyldes en kombination af at kvinder oftere s{\o}ger l{\ae}gehj{\ae}lp 
og det forhold at de har mindre risiko for at d{\o} i forbindelse med helbredsforv{\ae}rring.


Det tredje studie unders{\o}gte om gennemsnitsalderen ved f{\o}rste hospitalsind- l{\ae}ggelse 
efter det fyldte 60 {\aa}r var steget med tiden blandt danske m{\ae}nd og kvinder, og 
om denne tendens var ledsaget af en stigende eller faldende variation i gennemsnitsalderen. 
Resultaterne viser at sygdomme indtr{\ae}ffer senere i livet. Alligevel viser den stigende 
variation i alder ved f{\o}rste indl{\ae}ggelse, at det ikke er alle, der oplever denne 
sygdomsudskydelse.

Det fjerde studie omhandlede den nuv{\ae}rende demografiske profil for hospitaliseringer 
i Danmark samt en fremskrivning til {\aa}r 2050. Stigende levetid for m{\ae}nd og kvinder 
sammenholdt med m{\ae}nds h{\o}jere risiko for at blive indlagt efter den reproduktive 
alder betyder at m{\ae}nd p{\aa} 70+ forventes at blive den hurtigst voksende patientgruppe.

Med hensyn til hospitalsindl{\ae}ggelser har kvinder efter den reproduktive alder 
en r{\ae}kke fordele sammenlignet med m{\ae}nd: Kvinder har mindre risiko for at blive 
indlagt p{\aa} et hospital og de har lavere d{\o}delighed efter hospitalisering sammenlignet 
med m{\ae} nd. En mulig forklaring p{\aa} dette kan v{\ae}re at kvinder s{\o}ger l{\ae}ge 
tidligere i sygdomsforl{\o}bet. De forskellige landes sundhedssystemer arbejder hen mod 
at kunne rumme en aldrende befolkning, og de m{\aa} derfor v{\ae}re opm{\ae}rksomme p{\aa} 
disse k{\o}nsforskelle for at sikre at hospitalerne kan im{\o}dekomme fremtidige 
patienters behov.


%-----------------------------------------------------------------------------------------%
%-----------------------------------------------------------------------------------------%
