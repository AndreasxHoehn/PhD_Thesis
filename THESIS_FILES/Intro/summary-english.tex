

%-----------------------------------------------------------------------------------------%
%-----------------------------------------------------------------------------------------%


\chapter{Summary -- English}

Although women tend to have poorer health than men, they have 
a mortality advantage. This finding has been documented across 
different countries and contexts. However, the generalizability 
of this paradox is contested -- particularly among individuals of 
post-reproductive ages, when health deteriorates and aging accelerates. 
Routinely-collected hospital records may provide a unique framework 
for studying gender differences in mortality and health. These 
data cover information on medically diagnosed conditions, can often 
be linked with mortality data, and exist for a variety of countries, 
including Denmark. 

Using Danish register data, four papers were developed to analyze 
particular features of gender differences in mortality and health 
surrounding hospital admissions. In the first paper, male excess 
mortality following the first all-cause and cause-specific hospitalization 
after age 50 was estimated. Results from the first paper indicate 
that part of the female mortality advantage at post-reproductive 
ages can be attributed to their lower risk of dying following the 
deterioration of health.

The second paper examined how gender patterns in treatment-seeking 
behavior changed around a major health shock, measured as the first 
admission to hospital after age 60 for either stroke, myocardial 
infarction, chronic obstructive pulmonary disease, or gastrointestinal 
cancers. Findings point towards the fact that women's higher levels 
of primary healthcare use are attributable to a combination of both, 
a lower threshold to seek medical advice, and a health disadvantage 
resulting from lower mortality in bad health.

The third paper studied whether the mean age at first hospital admission 
after age 60 has increased among Danish men and women over time, and 
whether this trend has been accompanied by increasing or decreasing 
variation in the mean age. Results show that morbidity has been shifted 
towards older ages. However, increasing variation in the age at first 
admission indicates that this average increase has not been experienced 
by everyone.

In the fourth paper, the current demographic profile of hospital care 
use in Denmark was described, and changes up to 2050 were projected. 
Increasing longevity among men and women, in combination with men's 
higher risk to be hospitalized at post-reproductive ages, mean men 
aged 70+ are projected to be the fastest-growing patient group.

Regarding hospital admissions, women at post-reproductive ages have 
multiple advantages in comparison to men: Women have a lower risk of 
being admitted to hospital, and have lower mortality following 
hospitalization. One potential explanation for these patterns could 
be that women seek advice earlier in the disease course. Healthcare 
systems preparing for population aging should be aware of these gender 
patterns to ensure that hospitals meet the needs of future patients.






%-----------------------------------------------------------------------------------------%
%-----------------------------------------------------------------------------------------%
